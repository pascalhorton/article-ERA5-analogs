%%%%%%%%%%%%%%%%%%%%%%%%%%%%%%%%%%%%%%%%%%%%%%%%%%%%%%%
% A template for Wiley article submissions.
% Developed by Overleaf. 
%
% Please note that whilst this template provides a 
% preview of the typeset manuscript for submission, it 
% will not necessarily be the final publication layout.
%
% Usage notes:
% The "blind" option will make anonymous all author, affiliation, correspondence and funding information.
% Use "num-refs" option for numerical citation and references style.
% Use "alpha-refs" option for author-year citation and references style.

\documentclass[alpha-refs]{wiley-article}

% Add additional packages here if required
\usepackage{multirow}
\usepackage{gensymb}

\usepackage[modulo]{lineno}
\usepackage{setspace}
\onehalfspacing

%\geometry{margin=2cm}

% Guidelines: https://rmets.onlinelibrary.wiley.com/hub/journal/10970088/about/author-guidelines

% Update article type if known
\papertype{Research Article}

\title{Should analog methods use ERA5 for statistical precipitation downscaling? }

% List abbreviations here, if any. Please note that it is preferred that abbreviations be defined at the first instance they appear in the text, rather than creating an abbreviations list.
\abbrevs{AMs, analog methods; NWP, Numerical weather prediction; CP, calibration period; VP, validation period; AP, archive period}

% Include full author names and degrees, when required by the journal.
% Use the \authfn to add symbols for additional footnotes and present addresses, if any. Usually start with 1 for notes about author contributions; then continuing with 2 etc if any author has a different present address.
\author[1]{Pascal Horton}

% Include full affiliation details for all authors
\affil[1]{Oeschger Centre for Climate Change Research and Institute of Geography, University of Bern, Bern, Switzerland}

\corraddress{Pascal Horton, Oeschger Centre for Climate Change Research and Institute of Geography, University of Bern, 3012 Bern, Switzerland}
\corremail{pascal.horton@giub.unibe.ch}

%\presentadd[\authfn{2}]{Department, Institution, City, State or Province, Postal Code, Country}

%\fundinginfo{Funder One, Funder One Department, Grant/Award Number: 123456, 123457 and 123458; Funder Two, Funder Two Department, Grant/Award Number: 123459}

% Include the name of the author that should appear in the running header
\runningauthor{P. Horton}

\begin{document}

\maketitle

\begin{abstract}
Perfect prognosis statistical downscaling relies on statistical relationships established between observational data for both predictands and predictors. Predictors are often retrieved from reanalyses, which are considered as pseudo-observations. The impact of the choice of a reanalysis dataset on the skill of the downscaling method is usually overlooked as global reanalyses are frequently assumed to be equivalent for the last decades and data-rich regions such as Europe. However, it was recently shown that the reanalysis dataset has an impact on the method skill that can be even higher than the choice of the predictor variables. Generally, reanalyses that are processed by more recent atmospheric models and that assimilate more data perform best.

Following the recent release of ERA5, this work aimed at assessing the extent of the potential gains compared to other global reanalyses, including its predecessor, ERA-interim. The assessment was carried out using six variants of analog methods, which are statistical downscaling techniques, to predict the daily precipitation at 301 stations in Switzerland. ERA5 ended up among the best performing reanalyses and proved the best choice for 50\% of the stations on average across the different analog methods.

However, ERA5 is delivered with a high spatial resolution (0.25\degree), which turned out not to be relevant for analog methods, and which can even be a trap for simple calibration techniques. Indeed, the domains over which the predictor fields are compared need to be optimized, and high-resolution grids come along with numerous sub-optimal local minima. Besides the risk of poorly-calibrated domains, the high resolution also requires much higher computational time for no gain in skill, as long as the predictors considered are at a synoptic scale.


% Please include a maximum of seven keywords
\keywords{reanalyses, ERA5, statistical downscaling, analog method, precipitation, Switzerland}

\end{abstract}

\linenumbers

\section{Introduction}

Analog methods (AMs), which are statistical downscaling techniques, aim at predicting local meteorological variables, often daily precipitation, based on large-scale predictors. AMs rely on the hypothesis that similar synoptic situations are likely to result in similar local effects, plus a certain variability that is not explained by the considered predictors \citep{Lorenz1969}. To account for this unexplained variability, an ensemble of analog situations is considered. It thus provides a statistical prediction in the form of an empirical conditional distribution made of the corresponding observed predictand values. Multiple variants of the AM exist, eventually with different structures and approaches, but mainly with different predictors.

AMs are most often developed in a perfect prognosis framework \citep{Rummukainen1997, Maraun2010}, where the relationship is calibrated between large-scale and local-scale observations. In this context, global reanalyses are the datasets of choice for the large-scale variables as they provide multivariate gridded outputs that are physically consistent and available all around the globe \citep{Gelaro2017}. Globally, two main types of reanalysis products exist: those that aim for homogeneity over a long period -- starting at the beginning of the 20th century -- and hence assimilate surface data only, and those that aim for accuracy over a more recent period and therefore assimilate a maximum of observations, including multiple satellite products. The reanalyses accuracy depends on the quality of the model physics and the analysis process, thus on the quantity and quality of the assimilated observations \citep{Dee2011a}.

The choice of a reanalysis dataset to be used in AMs can be driven by the context of the application. For example, if it needs to cover the 20th century, ECMWF twentieth century reanalyses \citep[ERA-20C or CERA-20C --][]{Poli2016, Laloyaux2016} or the Twentieth Century Reanalysis \citep[20CR --][]{Compo2011} produced by NOAA can be used \citep[for example,][]{Kuentz2015, Caillouet2016, Brigode2016, Bonnet2017}. In an operational context, with the target situation being provided by an NWP model, one should prefer a reanalysis that is produced by the same NWP model as the operational forecast to reduce the inter-models biases. One might also prefer a sub-optimal reanalysis offering a longer archive, thus containing more candidate analogs. However, in many cases, the selection of the reanalysis is arbitrary and might be driven by a preference for the local provider.  

NCEP/NCAR Reanalysis 1 \citep[NR-1 --][]{Kalnay1996, Kistler2001} and NCEP/DOE Reanalysis 2 \citep[NR-2 --][]{Kanamitsu2002} were used in many applications of AMs \citep{Timbal2003, Bontron2004, Wetterhall2005a, Gangopadhyay2005, Altava-Ortiz2006, Barrera2007, Cannon2007, Matulla2007, Bliefernicht2007, Maurer2008, BenDaoud2009, Wu2012, Marty2012, Teng2012, Horton2012, Yiou2014}. ERA-40 \citep{Uppala2005}, produced by ECMWF, has also been used substantially \citep {BenDaoud2009, Willems2011b, JakobThemessl2011a, BenDaoud2011, Turco2011a, Franke2011, Pascual2012b, Schenk2012, Ribalaygua2013a, Osca2013, Radanovics2013, Martin2014b, Chardon2014, BenDaoud2016}. Its sucessor, ERA-Interim \citep[ERA-INT --][]{Dee2011a}, has been used by \cite{Raynaud2016b}. NASA's Modern-Era Retrospective Analysis for Research and Applications \citep[MERRA -- ][]{Rienecker2011} has been used by \citet{Vanvyve2015}. The Japanese products, the Japanese 55-year Reanalysis \citep[JRA-55 --][]{Kobayashi2015, Harada2016} and its conventional-only JRA-55 Conventional \citep[JRA-55C --][]{Kobayashi2014}, were not used in AMs to the author knowledge. Newer products, such as NCEP's the Climate Forecast System Reanalysis \citep[CFSR --][]{Saha2010a} or MERRA version 2 \citep[MERRA-2 -- ][]{Gelaro2017} are not much used yet in AMs. One can often observe a lag between the release of a new reanalysis dataset and its adoption in AMs.

Most applications of AMs are based on a single reanalysis dataset, and the impact of this choice is overlooked. \citet{BenDaoud2009} compared NR-1 to ERA-40 and found no significant difference for the predictors considered. Later, \citet{Dayon2015} compared NR-1, MERRA, ERA-INT, and 20CR and found out that the choice of the reanalysis dataset has a non-negligible impact on the performance of the AMs that can even be greater than the choice of the predictor variables. They concluded that the role of the reanalyses should not be underestimated. Such influence has also been observed for other statistical downscaling methods \citep[e.g.][]{Koukidis2009}. \citet{Horton2018b} compared ten global reanalyses and concluded that the impact of the dataset on the results of the AM is significant. As \citet{Dayon2015}, the impact of the reanalysis was found to be sometimes higher than that of the selected predictors. Some recommendations were established to select an appropriate reanalysis depending on the period and the predictors of interest. Globally, more recent products that assimilate more data show better skills \citep{Horton2018b}. In that respect, the recent release of ERA5 \citep{Hersbach2019} raises the question of its potential benefits for AMs compared to its predecessor ERA-Interim.

ERA5 was compared to nine other global reanalyses for seven AMs at 301 stations in Switzerland. The data and methods are described in section \ref{sec:data_methods}. The impact on the skill is presented in section \ref{sec:results_skill}, the spatial resolution is assessed in section \ref{sec:results_hires}, and the similarity to other reanalyses is presented in section \ref{sec:results_shared_dates}. The conclusions (section \ref{sec:conclusion}) summarize the key findings.


\section{Data and methods}
\label{sec:data_methods}

\subsection{Reanalyses}
\label{sec:reanalyses}

The present work aims at comparing ERA5 \citep{Hersbach2019} to other global reanalyses, which characteristics are provided in Table \ref{table:datasets}. The three first reanalyses are surface-input \citep{Fujiwara2017} products that assimilate surface data only but cover an extended period, typically the 20th century. NOAA produced the Twentieth Century Reanalysis \citep[version 2c, 20CR-2c --][]{Compo2011}, which only assimilates surface pressure data and uses observed monthly sea-surface temperature and sea-ice distributions as boundary conditions. The European Centre for Medium-Range Weather Forecasts (ECMWF) developed two products using surface-input only. The first is ERA-20C \citep{Poli2016}, which assimilates marine wind observations and is forced by sea surface temperature, sea ice cover, atmospheric composition changes, and solar forcing. The second is CERA-20C \citep{Laloyaux2018a}, which has an additional coupling to the ocean and was produced by a more recent version of the IFS model.

The other reanalyses are full-input products that assimilate all available data, including satellite data \citep{Fujiwara2017}. NCEP/NCAR Reanalysis I \citep[NR-1 --][]{Kalnay1996, Kistler2001} was the first global reanalysis, followed by NCEP/DOE Reanalysis 2 \citep[NR-2 --][]{Kanamitsu2002} that fixed some identified problems. The Climate Forecast System Reanalysis \citep[CFSR --][]{Saha2010a} is the most recent reanalysis by NCEP. The Japanese 55-year Reanalysis \citep[JRA-55 --][]{Kobayashi2015, Harada2016} is produced by the Japan Meteorological Agency (JMA). Its version using conventional data only, JRA-55 Conventional \citep[JRA-55C --][]{Kobayashi2014}, was not considered in this work as it provides similar results as JRA-55 \citep{Horton2018b}. NASA's Global Modeling and Assimilation Office (GMAO) released the Modern-Era Retrospective Analysis for Research and Applications, version 2 \citep[MERRA-2 -- ][]{Gelaro2017}, which is an improvement of the first MERRA reanalysis \citep{Rienecker2011}. Finally, the predecessor of ERA5, ERA-Interim \citep[ERA-INT --][]{Dee2011a}, was also considered.

ERA5 \citep{Hersbach2019} is meant to replace ERA-Interim. It benefits from multiple improvements to the Integrating Forecasting System (IFS), in terms of model physics, core dynamics, and data assimilation \citep{Hersbach2019}. It provides more outputs, with higher temporal (hourly) and spatial (0.28\degree; 31~km) resolutions. The use of a 10-members ensemble of data assimilations at a lower spatial and temporal resolution allows an estimation of the uncertainty. ERA5 assimilates significantly more data than ERA-Interim, such as ground-based radar and new satellite sensors, and uses improved observation operators, which allows better comparing model outputs to observations. Additionally, it indirectly benefits from improvements of historical observations, both for conventional and satellite data \citep{Hersbach2019}. ERA5 provides, among others, more consistent sea surface temperature and sea ice, an improved representation of tropical cyclones, a better balance of evaporation and precipitation, and improved soil moisture. ERA5 is also more suitable for climate analyses as it relies on appropriate radiative forcing and boundary conditions (e.g., changes in greenhouse gases, aerosols, SST, and sea ice).


\begin{table}[bt]
	\caption{Assessed reanalysis datasets with their respective properties, sorted by type and model age.}
	\small
	\begin{threeparttable}
		\begin{tabular}{lllllll}
			\hline
			\headrow
			\thead{Name} & \thead{Institution} & \thead{Coverage} & \thead{Output} & \thead{Model resolution \& age} & \thead{Input} & \thead{Assimilation}\\
			\hline 
			\textbf{20CR-2c} & NOAA-CIRES & 1851 -- 2014 & 2\degree x 2\degree & T62 ($\sim$1.88\degree), L28, 2008 & surface  & EnKF\\
			\textbf{ERA-20C} & ECMWF & 1900 -- 2010 & 1\degree x 1\degree & TL159 ($\sim$1.13\degree), L9, 2012 & surface  & 4D-Var\\
			\textbf{CERA-20C} & ECMWF & 1901 -- 2010 & 1\degree x 1\degree & T159 ($\sim$1.13\degree), L91, 2016 & surface & 4D-Var\\
			\hline 
			\textbf{NR-1} & NCEP, NCAR & 1948 -- present & 2.5\degree x 2.5\degree & T62 ($\sim$1.88\degree), L28, 1995 & full & 3D-Var\\
			\textbf{NR-2} & NCEP, DOE & 1979 -- present & 2.5\degree x 2.5\degree & T62 ($\sim$1.88\degree), L28, 2001 & full  & 3D-Var\\
			\textbf{CFSR} & NCEP & 1979 -- present & 0.5\degree x 0.5\degree & T382 ($\sim$0.31\degree), L64, 2009 & full  & 3D-Var\\
			\textbf{JRA-55}  & JMA & 1958 -- present & 1.25\degree x 1.25\degree & TL319 ($\sim$0.36\degree), L60, 2009 & full  & 4D-Var\\
			%\textbf{JRA-55C}  & JMA & 1958 -- 2015 & 1.25\degree x 1.25\degree & TL319 ($\sim$0.36\degree), L60 & 2009 & conventional  & 4D-Var\\
			\textbf{MERRA-2} & NASA GMAO & 1980 -- present & 0.625\degree x 0.5\degree & 0.625\degree x 0.5\degree, L72, 2014 & full  & 3D-Var\\
			\textbf{ERA-INT} & ECMWF & 1979 -- 2017 & 0.75\degree x 0.75\degree & TL255 ($\sim$0.70\degree), L60, 2006 & full  & 4D-Var\\
			\textbf{ERA5} & ECMWF & 1950* -- present & 0.25\degree x 0.25\degree & T639 ($\sim$0.28\degree), L137, 2016 & full  & 4D-Var\\
			\hline 
		\end{tabular} 
		
		\begin{tablenotes}
			\item *expected to be available in 2020. The available period starts in 1979 at time of writing.
			%\item JKL, just keep laughing; MN, merry noise.
		\end{tablenotes}
	\end{threeparttable}
	\label{table:datasets}
\end{table}


\subsection{Precipitation dataset}
\label{sec:precipitation}

The local variables to be predicted are here daily precipitation totals (06:00~h~UTC to 06:00~h~UTC the following day) at 301 stations of the MeteoSwiss network in Switzerland (Fig. \ref{fig:stations}), with good coverage of the 1981--2010 period. The 30-year precipitation dataset was divided into a calibration period (CP) and an independent validation period (VP), which was evenly distributed over the entire series (1 year out of every 5; a total of 6 years). The archive period (AP), where the analog dates are being retrieved, is the same as the CP -- also with the VP being excluded -- with an additional exclusion of $\pm30$ days around the target date. The results below are always presented for the VP.


\subsection{Analog methods}
\label{sec:ams}

Most AMs variants from \cite{Horton2018b} were considered here (Table \ref{table:methods}). These AMs have different degrees of complexity, but they all start with a preselection in order to cope with seasonality. A commonly used preselection is based on the dates (PC: preselection on a calendar basis in Table \ref{table:methods}) as candidate situations are extracted from the archive for a period of maximum 120~days centered around the target date. To allow for a more dynamic approach, \citet{BenDaoud2016} base this preselection on the similarity of air temperature (T) at 925~hPa and 600~hPa at the nearest grid point. A resulting undesired mixing of spring and autumn situations is discussed in \citet{Caillouet2016}. 

\begin{table*}[t]
	\caption{Analogue methods considered in the study, listed by increasing complexity. The analogy criterion is S1 for SLP and Z and RMSE for the other variables.}
	\small
	\begin{threeparttable}
		\begin{tabular}{llllll}
			\hline
			\headrow
			\thead{Method} & \thead{P0} & \thead{L1} & \thead{L2} & \thead{L3} & \thead{Reference} \\ 
			\hline 
			\multirow{2}{*}{\textbf{2Z}} & \multirow{2}{*}{PC} & Z1000@12h &&& \multirow{2}{*}{\citealp{Bontron2004}} \\
			&& Z500@24h &&& \\
			\hline 
			\multirow{4}{*}{\textbf{4Z}} & \multirow{4}{*}{PC} & Z1000@06h &&& \multirow{4}{*}{\citealp{Horton2018a}} \\
			&& Z1000@30h &&& \\
			&& Z700@24h &&& \\
			&& Z500@12h &&& \\
			\hline 
			\multirow{2}{*}{\textbf{2Z-2MI}} & \multirow{2}{*}{PC} & Z1000@12h & \multirow{2}{*}{MI850@12+24h} && \multirow{2}{*}{\citealp{Bontron2004}} \\
			&& Z500@24h &&& \\
			\hline 
			\multirow{4}{*}{\textbf{4Z-2MI}} & \multirow{4}{*}{PC} & Z1000@30h &&& \multirow{4}{*}{\citealp{Horton2018a}}\\
			&& Z850@12h & MI700@24h && \\
			&& Z700@24h & MI600@12h && \\
			&& Z400@12h &&& \\
			\hline 
			\multirow{2}{*}{\textbf{PT-2Z-4MI}} & T925@36h & Z1000@12h & MI925@12+24h && \multirow{2}{*}{\citealp{BenDaoud2016}} \\
			& T600@12h & Z500@24h & MI700@12+24h && \\
			\hline 
			\multirow{2}{*}{\textbf{PT-2Z-4W-4MI}} & T925@36h & Z1000@12h & \multirow{2}{*}{W850@06-24h} & MI925@12+24h & \multirow{2}{*}{\citealp{BenDaoud2016}} \\
			& T600@12h & Z500@24h && MI700@12+24h & \\
			\hline 
		\end{tabular} 
		
		\begin{tablenotes}
			\item P0, preselection (PC: $\pm 60$ days around the target date); L1, L2 and L3, subsequent levels of analogy.
			\item Z, geopotential height; T, air temperature; W, vertical velocity; MI, moisture index (product of the relative humidity at the given pressure level and the total water column).
		\end{tablenotes}
	\end{threeparttable}
	\label{table:methods}
\end{table*}

All considered AMs have a first level of analogy on the atmospheric circulation, using the geopotential height (Z) at different pressure levels and time as predictor. This analogy is quantified by the S1 criterion \citep[Eq.\ \ref{eq:S1}, ][]{Teweles1954, Brown2012}, which is a comparison of gradients over a selected (calibrated) domain. Its purpose is to compare the atmospheric flow and not the absolute values of the respective fields. The values of the criterion processed for different levels/hours are here averaged into a single value for a given candidate date. More advanced approaches introduce a weighting between these diverse components \citep{Horton2017a}.

\begin{equation}
	\label{eq:S1}
	S1=100 \frac{\sum_{i} \vert \Delta\hat{z}_{i} - \Delta z_{i} \vert}{\sum_{i} \max\left\lbrace \vert \Delta\hat{z}_{i} \vert; \vert \Delta z_{i} \vert \right\rbrace }
	% use \displaystyle to max sum sign larger
\end{equation}
where $\Delta \hat{z}_{i}$ is the geopotential height gradient between the \textit{i}-th pair of points for the target day, and $\Delta z_{i}$ is the corresponding observed geopotential height gradient for the candidate situation. The smaller the value S1 is, the more similar the pressure fields.

The most simple method, 2Z, is based only on the analogy of the atmospheric circulation, using the geopotential height at 1000~hPa and 500~hPa \citep{Bontron2004}. The most similar $N_{1}$ dates with the lowest values of S1 are selected as analogs to the target date. The daily precipitation that was measured at these $N_{1}$ dates then provides the empirical conditional distribution, considered as the probabilistic prediction for the target date.

The second method, 4Z, is also based on geopotential heights only, but using four combinations of pressure levels and temporal windows \citep{Horton2018a}. This method is here considered as a simplified version of a more sophisticated method elaborated by genetic algorithms. It was found that using four geopotential heights is more informative than using two.

Then, the next three methods (2Z-2MI, 4Z-2MI, PT-2Z-4MI, Table \ref{table:methods}) add a second level of analogy based on moisture variables to subsample from the first level. The predictor considered here is a moisture index (MI), which is the product of the total precipitable water (TPW) and the relative humidity (RH) and was introduced by \citet{Bontron2004}. This predictor is assessed using the root mean square error (RMSE) criterion. The 2Z-2MI method \citep{Bontron2004} uses this index at 850~hPa for two different hours (+12 h and +24 h). The 4Z-2MI method \citep{Horton2018a}, derived from optimizations with genetic algorithms, uses the moisture index at 600~hPa and 700~hPa, with a change in the selected levels of the geopotential height compared to 4Z. The 4Z-2MI method \citep{BenDaoud2016} considers the moisture index at 700~hPa and 925~hPa, both at +12 h and +24 h.

Finally, the most complex method, PT-2Z-4W-4MI, also named "SANDHY" for Stepwise Analogue Downscaling method for Hydrology \citep{BenDaoud2016, Caillouet2016}, adds an intermediate level of analogy, before the moisture predictors, based on the vertical velocity (W) at 850~hPa. It was primarily developed for large and relatively flat/lowland catchments in France (Sa\^{o}ne, Seine).


\subsection{Calibration of the methods} %\subsubsection?
\label{sec:calibration}

The CRPS \citep[Continuous Ranked Probability Score,][]{Brown1974, Matheson1976, Hersbach2000} is often used as the performance metric for AMs as it allows evaluating the predicted cumulative distribution provided by the analogs with regards to the single observed value at the target date. Its skill score expression, the CRPSS (Continuous Ranked Probability Skill Score), was used here, using the climatological distribution of precipitation at each station as the reference. It allows a better comparison between stations as it takes into account the differences in climatology to some extent. The CRPSS is expressed as follows \citep{Bradley2011}:

\begin{equation}
	\label{eq:CRPSS}
	CRPSS = 1-\frac{\overline{CRPS}}{\overline{CRPS}_{clim}}
\end{equation}
where $CRPS_{clim}$ is the CRPS value for the climatological distribution. A better prediction is characterized by an increase in CRPSS.

The methods were here calibrated using the semi-automatic sequential procedure developed by \citet{Bontron2004} and also used in \citet{Radanovics2013} and \citet{BenDaoud2016}. A detailed description can also be found in \citet{Horton2019}. It allows calibrating the spatial windows on which the predictors are compared (specific to each level of analogy) and the number of analog dates to select for the different levels of analogy. The AtmoSwing software \citep{Horton2019} was used to perform this calibration independently for every station, method, and reanalysis. The calculations were performed on an HPC cluster at the University of Bern. 

After the initial calibration of all AMs for ERA5, some issues were identified with the spatial windows. These were due to some limitations of the calibration procedure related to the high-resolution grid. The process was then restarted using an improvement of the semi-automatic procedure, named classic+ \citep{Horton2019}, which allows additional moves in the calibration of the spatial windows and thus allows escaping local minima. This issue is shown and discussed in section \ref{sec:results_hires}.


\section{Results}
\label{sec:results}

The results are provided for the VP (independent validation period, section \ref{sec:precipitation}). All reanalyses were used at their highest available spatial resolution and at a 6-hourly time step. The impact of the spatial resolution is analyzed in section \ref{sec:results_hires}.

\subsection{Impact on the skill}
\label{sec:results_skill}

The skill scores of the different methods and reanalyses are shown in Fig. \ref{fig:comparison_values}. As discussed in \citet{Horton2018b}, there are clear differences between methods with more complex AMs generally performing better, and also between reanalyses in favor of modern full-input reanalyses. One can see here that ERA5 is systematically in the best performing reanalyses. When isolating the impact of the reanalysis by removing the influence of the methods and the stations (Fig. \ref{fig:comparison_relative}), the positive contribution of ERA5 to the skill score is even more clear, with the largest difference for the method PT-2Z-4MI. One can also see a clear improvement compared to ERA-INT.

ERA5 was selected as the best reanalysis for about 50\% of the stations on average, with a minimum of 34.9\% for 4Z and a maximum of 78.7\% for PT-2Z-4MI. When considering the first two ranks, ERA5 was selected 75\% on average, with a minimum of 61.8\% for PT-2Z-4W-4MI and a maximum of 91.7\% for PT-2Z-4MI. There were no specific spatial patterns as to which stations performed best with ERA5, such as it was the case with other reanalyses in \citet{Horton2018b}. Also, ERA-INT was not systematically superseded by ERA5, and ERA-INT remained the best choice for some stations and methods. It is thus important to undertake such a comparison with different AMs and multiple stations.


\subsection{High resolution and its pitfalls}
\label{sec:results_hires}

Although the calibrations have been made on an HPC cluster, they ended up to be highly time-consuming when using the semi-automatic sequential procedure (classic calibration, Sect. \ref{sec:calibration}). The reason is that the process involves first the calculation of a relevance map for each level of analogy, which requires an assessment of every unitary cell of the predictor grid. Then, the spatial window in growing by assessing extensions in every four dimensions and by applying the one with the highest gain. On a high-resolution grid, these procedures imply numerous assessments.

The first results of the calibration showed that ERA5 (grayed bars in Fig. \ref{fig:resolution}) underperformed ERA-INT. An examination of these results revealed that the spatial windows did not grow as much as expected and were not optimal. The reason being that the classic calibration got stuck in local minima during the procedure of the spatial window expansion, as none of the increments in the four directions resulted in a gain in skill. This issue did not happen in a significant way for the other reanalyses and is here due to the high resolution of ERA5. The high resolution generates in a noisier response surface in terms of skill, with local minima that trap the classic calibration. To address this issue, the classic+ calibration (Sect. \ref{sec:calibration}) was then used. It allows additional moves in the growth of the spatial windows, including increases of multiple grid points at once to get out of local minima. The classic+ calibration also ended up to be faster as it does not assess every grid point in the calculation of the relevance map. A global optimization technique, such as genetic algorithms \citep{Horton2017a}, would also efficiently overcome this issue. However, they are generally computationally intensive and too much time consuming for this comparison work. With the advent of high-resolution datasets, one should be aware of the potential pitfalls of calibration techniques that were developed on low-resolution datasets

\citet{Horton2018b} showed that below 1\degree, the spatial resolution of the reanalyses generally has no significant impact on the performance of AMs. It is because, e.g., the geopotential at 500~hPa has a half-autocorrelation distance of about 1000~km \cite{Thiebaux1985}, and thus, a high resolution does not bring new information. It might, however, be different for moisture variables. ERA5 has the highest spatial resolution among global reanalyses. As the results in Sect. \ref{sec:results_skill} were obtained using the full spatial resolution (0.25\degree), an additional calibration was here performed using the same resolution as ERA-INT (0.75\degree) to quantify the contribution of the resolution to the gain in skill. Fig. \ref{fig:resolution} shows that degrading the resolution of ERA5 to ERA-INT resolution did not significantly impact the score. Although a higher model resolution is likely contributing to the improvements of ERA5 compared to ERA-INT, the high output resolution has no impact on AMs performance for the prediction of daily precipitation. We would have expected to see some gains for moisture variables, but they did not materialize. The gains are thus likely due to improvements in the forecast model, the assimilation scheme, and the assimilated data. While this conclusion is likely valid for predictands that are principally driven by large-scale predictors, it cannot be transposed to all applications. Indeed, when the phenomenon at hand is mainly driven by mesoscale predictors, such as lightning activity, the output spatial resolution does matter (unpublished results).


\subsection{Shared analog dates}
\label{sec:results_shared_dates}

The use of predictors from different datasets, and with distinct spatial windows, results in a different selection of analog dates. The percentage of identical analog dates for the different AMs over the VP is shown in Fig. \ref{fig:shared-dates}. This percentage decreases with the complexity of the AM, and range from a maximum of 74\% between NR-1 and NR-2 for the 2Z method to a minimum of 10\% between MERRA-2 and 20CR-2c for the PT-2Z-4W-4MI method. It is likely that if all reanalyses were compared over the same spatial windows, these numbers would be higher.

As it could be expected, ERA5 shared some similarities in the selection of analog dates with ERA-INT, but not to an extent that is significantly different from the other modern full-input reanalyses. Indeed, the percentage of shared analog dates is of the same order of magnitude as with CFSR and JRA-55, which are the two reanalyses produced by models with a high spatial resolution (Table \ref{table:datasets}), but with fewer vertical levels. It is worth noting that JRA-55, which presents here strong similarities with ERA5 in terms of selected analog dates for Switzerland, has an output resolution that is five times lower (1.25\degree). It should be noted, however, that JRA-55 shares more analog dates with ERA-INT than with ERA5. 


\section{Conclusions}
\label{sec:conclusion}

The choice of a reanalysis has a significant impact on the performance of AMs in the perfect prognosis context, as already shown in \citet{Horton2018b}. The impact of the dataset on AMs skills can be higher than the one related to the choice of the predictor variables. Other authors drew the same conclusions \cite{Dayon2015}. Generally, the latest version of the full-input reanalyses should be used as they contribute to a better skill for Switzerland, which is also likely valid for Europe. The subsequent release of ERA5 then raised the question of its potential benefits over ERA-INT, and to what extent the high spatial resolution could be informative for statistical downscaling.

This work focused on comparing ERA5 to the other main global reanalyses and showed that it should be the dataset of choice for all tested AMs as it often surpassed the other reanalyses. ERA5 was selected as the best dataset by about 50\% of the stations and was in one of the two first ranks for 75\% of the stations. Although ERA5 globally performed better than ERA-INT for the assessed AMs, ERA-INT was not systematically superseded by ERA5 for all stations. It is thus important to undertake such comparison with a large enough sample of stations in order to capture well the overall picture. Even though ERA5 might not be the best reanalysis for all stations, it would not be advisable to patchwork the selection of reanalyses among stations. It would be recommendable, however, to use not a single but a selection of reanalyses for some applications.

ERA5 is provided at high spatial (0.25\degree) and temporal (1~h) resolutions. Although AMs are computationally-light techniques, their calibration over several decades can be time-consuming. This time increases all the more with a higher spatial resolution of the predictor grids. The high resolution even turned into a trap for the calibration technique here that ended up stuck in local minima, resulting in poor skills. When using high-resolution predictor grids, one should take care that the calibration technique used can satisfactorily optimize the spatial windows. The classic+ calibration \citep{Horton2019} could here successfully calibrate the spatial windows thanks to additional moves that allow going out of local minima. 

\citet{Horton2018b} showed that below 1\degree, the gains are limited for AMs. ERA5 was here used at its original resolution as well as with the same resolution as ERA-INT (0.75\degree). The reduction in resolution did not impact the performance of AMs for daily precipitation, here mainly driven by synoptic-scale processes. This conclusion was shown to be not valid for predictands driven by mesoscale drivers. It also implies that the output resolution does not explain the gain in skill compared to ERA-INT, which are likely due to improvements in the forecast model, the assimilation scheme, and the assimilated data.

To conclude, the use of ERA5 is recommended for the prediction of daily precipitation with AMs in Europe, ideally along with other reanalyses to provide a multi-model approach. However, for predictands that are primarily driven by synoptic-scale processes, using ERA5 at its full resolution is not justified and might even be counterproductive if the calibration method does not handle it well.



\section*{acknowledgements}
Precipitation time series were provided by MeteoSwiss. The NCEP/NCAR, NCEP/DOE, and 20CR-2c were provided by the NOAA/OAR/ESRL PSD, Boulder, Colorado, USA, at http://www.esrl.noaa.gov/psd/. Support for the Twentieth Century Reanalysis Project dataset is provided by the US Department of Energy, Office of Science Innovative and Novel Computational Impact on Theory and Experiment (DOE INCITE) program, and Office of Biological and Environmental Research (BER), and by the National Oceanic and Atmospheric Administration Climate Program Office. The CFSR and JRA-55 were obtained from the CISL Research Data Archive (http://rda.ucar.edu/) at NCAR in Boulder, Colorado, and the NCAR is supported by grants from the National Science Foundation. The Climate Forecast System Reanalysis (CFSR) project is carried out by the Environmental Modeling Center (EMC), National Centers for Environmental Prediction (NCEP). The Japanese 55-year Reanalysis (JRA-55) project is carried out by the Japan Meteorological Agency (JMA). The MERRA-2 was obtained from the Goddard Earth Sciences Data and Information Services Center, Greenbelt, Maryland, from their website at http://disc.sci.gsfc.nasa.gov/mdisc. ERA-interim, ERA-20C, and CERA-20C were obtained from the ECMWF Data Server at http://apps.ecmwf.int/datasets/. ERA5 was obtained from the C3S climate data store (CDS) at https://cds.climate.copernicus.eu. Calculations were performed on UBELIX (http://www.id.unibe.ch/hpc), the HPC cluster at the University of Bern. Thanks to C. Obled for correcting the manuscript.


\section*{data availability}
All calculations were performed with the open-source AtmoSwing software v2.1.1 \citep{Horton2019c}. The resulting files were processed using AtmoSwing R-toolbox v1.2.0 \citep{Horton2018d}.

The resulting analog dates for every combination of station, reanalysis, and analog method were published. These archives also contain different files: the parameter files used in AtmoSwing for the calibration, the resulting calibrated parameters, and files listing all assessed parameter sets. These files are available for ERA5 (<...TO BE UPDATED...>) and the other reanalyses \citep[see references in][]{Horton2018b}.

%TODO: publish data


\section*{conflict of interest}
The author has declared no conflict of interest.


% Here are examples of quotes and epigraphs.
%\begin{quote}
%The significant problems we have cannot be solved at the same level of thinking with which we created them.\endnote{Albert Einstein said this.}
%\end{quote}

%\begin{epigraph}{Albert Einstein}
%Anyone who has never made a mistake has never tried anything new.
%\end{epigraph}

\linespread{1}

%\printendnotes

% Submissions are not required to reflect the precise reference formatting of the journal (use of italics, bold etc.), however it is important that all key elements of each reference are included.
\bibliography{references}



\graphicalabstract{figure-graph-abstract.pdf}{It was recently shown that reanalyses have an impact on statistical downscaling methods that can be even higher than the choice of the predictor variables. ERA5 proved here to be relevant when compared to other global reanalyses for the prediction of daily precipitation at 301 stations in Switzerland using six variants of analog methods. However, its high spatial resolution did not contribute to a gain in skill and was even found to be counterproductive for simple calibration techniques.}



\clearpage
\pagebreak

\begin{figure}
	\includegraphics[width=80mm]{figure-1.jpg}\\
	\caption{Map of the 301 precipitation stations with good data coverage of the period 1981--2010. Background map: \textcopyright\ SwissTopo.}
	\label{fig:stations}
\end{figure}

\begin{figure}[bt]
	\centering
	\includegraphics[width=\textwidth]{figure-2.pdf}
	\caption{CRPSS for all stations and all considered AMs and reanalysis datasets on the VP. A higher CRPSS means better performance. The parameters of the AMs were calibrated for every station, every dataset, and every method. The boxes show the 25th, 50th, and 75th percentiles. The whiskers extend to the most extreme data point, which is no more than 1.5 times the interquartile range.}
	\label{fig:comparison_values}
\end{figure}

\begin{figure}[bt]
	\centering
	\includegraphics[width=\textwidth]{figure-3.pdf}
	\caption{Impact of the reanalysis dataset on performance, isolated by processing the improvement in CRPSS for one dataset compared to the mean performance on all datasets, per station and method. Note that the methods cannot be compared here, only the datasets. Same conventions as Fig. 2.}
	\label{fig:comparison_relative}
\end{figure}

\begin{figure}[bt]
	\centering
	\includegraphics[width=100mm]{figure-4.pdf}
	\caption{Comparison of ERA-INT and ERA5 at 0.25\degree\ and 0.75\degree\ as well as a failed calibration of AMs with ERA5 (gray; see explanation in text). The impact of the reanalysis on performance has been isolated by processing the improvement in CRPSS for one dataset compared to the mean performance on all datasets (excluding the failed calibration), per station and per method. Same conventions as Fig. 2.}
	\label{fig:resolution}
\end{figure}

\begin{figure}[bt]
	\centering
	\includegraphics[width=120mm]{figure-5.pdf}
	\caption{Percentage of identical analog dates, for different AMs, selected when using the reanalyses in columns that were also found when using the reanalyses in rows. The values were averaged for all stations on the VP.}
	\label{fig:shared-dates}
\end{figure}


\end{document}
