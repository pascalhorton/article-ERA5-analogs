%%%%%%%%%%%%%%%%%%%%%%%%%%%%%%%%%%%%%%%%%%%%%%%%%%%%%%%
% A template for Wiley article submissions.
% Developed by Overleaf. 
%
% Please note that whilst this template provides a 
% preview of the typeset manuscript for submission, it 
% will not necessarily be the final publication layout.
%
% Usage notes:
% The "blind" option will make anonymous all author, affiliation, correspondence and funding information.
% Use "num-refs" option for numerical citation and references style.
% Use "alpha-refs" option for author-year citation and references style.

\documentclass[alpha-refs]{wiley-article}

% Add additional packages here if required
\usepackage{multirow}
\usepackage{gensymb}

% Update article type if known
\papertype{Original Article}
% Include section in journal if known, otherwise delete
%\paperfield{Journal Section}

\title{Should analog methods use ERA5 for statistical precipitation downscaling? }

% List abbreviations here, if any. Please note that it is preferred that abbreviations be defined at the first instance they appear in the text, rather than creating an abbreviations list.
\abbrevs{AMs, analog methods; ....}

% Include full author names and degrees, when required by the journal.
% Use the \authfn to add symbols for additional footnotes and present addresses, if any. Usually start with 1 for notes about author contributions; then continuing with 2 etc if any author has a different present address.
\author[1]{Pascal Horton}

% Include full affiliation details for all authors
\affil[1]{Oeschger Centre for Climate Change Research and Institute of Geography, University of Bern, Bern, Switzerland}

\corraddress{Pascal Horton, Oeschger Centre for Climate Change Research and Institute of Geography, University of Bern, 3012 Bern, Switzerland}
\corremail{pascal.horton@giub.unibe.ch}

%\presentadd[\authfn{2}]{Department, Institution, City, State or Province, Postal Code, Country}

%\fundinginfo{Funder One, Funder One Department, Grant/Award Number: 123456, 123457 and 123458; Funder Two, Funder Two Department, Grant/Award Number: 123459}

% Include the name of the author that should appear in the running header
\runningauthor{P. Horton}

\begin{document}

\maketitle

\begin{abstract}
This is a generic template designed for use by multiple journals, which includes several options for customization. Please consult the author guidelines for the journal to which you are submitting in order to confirm that your manuscript will comply with the journal's requirements. Please replace this text with your abstract.

% Please include a maximum of seven keywords
\keywords{keyword 1, \emph{keyword 2}, keyword 3, keyword 4, keyword 5, keyword 6, keyword 7}
\end{abstract}

\section{Introduction}

Analog methods (AMs), which are part of the statistical downscaling techniques, aim at predicting local meteorological variables, often daily precipitation, based on large-scale predictors. AMs rely on the hypothesis that similar synoptic situations are likely to result in similar local effects, plus a certain variability that is not explained by the considered predictors \citep{Lorenz1969}. To account for this unexplained variability, an ensemble of analog situations is considered, providing an statistical prediction in the form of an empirical conditional distribution made of the corresponding observed predictand values. Multiple variants of the AM exist, eventually with different structures and approach, but mainly with different selections of predictors.

AMs are most often developed in a perfect prognosis framework \citep{Rummukainen1997, Maraun2010}, where the relationship is calibrated between large-scale and local-scale observations. In this context, global reanalyses are the datasets of choice for the large-scale variables as they provide multivariate gridded outputs that are physically consistent and available all around the globe \citep{Gelaro2017}. There are globally two main types of reanalysis products: those that aim for homogeneity over a long period -- starting at the beginning of the 20th century -- and thus assimilate surface data only, and those that aim for accuracy over a more recent period and that assimilate a maximum of observations, including multiple satellite products. The accuracy of the reanalyses depends on the quality of the model physics and that of the analysis process, thus on the quantity and quality of the assimilated observations \citep{Dee2011a}.

The choice of a reanalysis dataset to be used in an AM can be driven by the context of the application, for example if it requires to cover the 20th century or to be used in an operational context, with the target situation being provided by a NWP model. In the first case, ECMWF twentieth century reanalyses \citep[ERA-20C or CERA-20C --][]{Poli2016, Laloyaux2016} or the Twentieth Century Reanalysis \citep[20CR --][]{Compo2011} produced by NOAA can be used \citep[for example,][]{Kuentz2015, Caillouet2016, Brigode2016, Bonnet2017}. In the second case, one should prefer a reanalysis that is produced by the same model as the operational forecast to reduce the inter-models biases. However, in many cases, the selection of the reanalysis is arbitrary and might be driven by a preference for the local provider. 

NCEP/NCAR Reanalysis 1 \citep[NR-1 --][]{Kalnay1996, Kistler2001} and NCEP/DOE Reanalysis 2 \citep[NR-2 --][]{Kanamitsu2002} were used in many applications of AMs \citep{Timbal2003, Bontron2004, Wetterhall2005a, Gangopadhyay2005, Altava-Ortiz2006, Barrera2007, Cannon2007, Matulla2007, Bliefernicht2007, Maurer2008, BenDaoud2009, Wu2012, Marty2012, Teng2012, Horton2012, Yiou2014}. ERA-40 \citep{Uppala2005}, produced by ECMWF, has also been used substantially \citep {BenDaoud2009, Willems2011b, JakobThemessl2011a, BenDaoud2011, Turco2011a, Franke2011, Pascual2012b, Schenk2012, Ribalaygua2013a, Osca2013, Radanovics2013, Martin2014b, Chardon2014, BenDaoud2016}. Its sucessor, ERA-Interim \citep[ERA-INT --][]{Dee2011a}, has been used by \cite{Raynaud2016b}. NASA's Modern-Era Retrospective Analysis for Research and Applications \citep[MERRA -- ][]{Rienecker2011} has been used by \citet{Vanvyve2015}. The Japanese products, the Japanese 55-year Reanalysis \citep[JRA-55 --][]{Kobayashi2015, Harada2016} and its conventional-only JRA-55 Conventional \citep[JRA-55C --][]{Kobayashi2014}, were not used in AMs to the author knowledge. Newer products, such as NCEP's the Climate Forecast System Reanalysis \citep[CFSR --][]{Saha2010a} or MERRA version 2 \citep[MERRA-2 -- ][]{Gelaro2017} are not much used yet in AMs. One can often observe a lag between the release of a new reanalysis dataset and its adoption in AMs.

Most applications of AMs are based on a single reanalysis dataset and the impact of this choice is overlooked. \citet{BenDaoud2009} compared NR-1 to ERA-40 and found no significant difference for the predictors considered. Later, \citet{Dayon2015} compared NR-1, MERRA, ERA-INT and 20CR and found out that the choice of the reanalysis dataset has a non-negligible impact on the performance of the AMs that can even be greater than the choice of the predictor variables. They concluded that the role of the reanalyses should not be underestimated. Such an influence has also been observed for other statistical downscaling methods \citep[e.g.][]{Koukidis2009}. \citet{Horton2018b} compared ten global reanalyses and concluded that the impact of the dataset on the results of the AM are significant. The conclusion was similar to that of \citet{Dayon2015}, in that the influence of the reanalysis can be bigger than that of the predictors selection. Some recommendations were established to guide for the choice of the reanalysis depending on the period and the predictors of interest. Globally, more recent products that assimilate more data show better skills \citep{Horton2018b}. In that respect, the recent release of ERA5 \citep{Hersbach2019} raises the question of how better it can be for AMs compared to its predecessor ERA-Interim, and what are the benefits and eventual pitfalls of its higher spatial and temporal resolutions.

ERA5 was compared to nine other global reanalyses for seven AMs at 301 stations in Switzerland. The data and methods are described in section \ref{sec:data_methods}. The impact of the skill is presented in section \ref{sec:results_skill}, the spatial resolution is assessed in section \ref{sec:results_hires}, and the similarity to other reanalyses is presented in section \ref{sec:results_shared_dates}. The conclusion (section \ref{sec:conclusions}) summarizes the key findings.


\section{Data and methods}
\label{sec:data_methods}

\subsection{Reanalyses}
\label{sec:reanalyses}

The present work aims at comparing ERA5 \citep{Hersbach2019} to other global reanalyses, which characteristics are provided in Table \ref{table:datasets}. The three first reanalyses are surface-input \citep{Fujiwara2017} products that assimilate surface data only, but cover a long period, typically the 20th century. NOAA produced the Twentieth Century Reanalysis \citep[version 2c, 20CR-2c --][]{Compo2011} which only assimilates surface pressure data and uses observed monthly sea-surface temperature and sea-ice distributions as boundary conditions. The European Centre for Medium-Range Weather Forecasts (ECMWF) developed two products using surface input only: ERA-20C \citep{Poli2016} that assimilates marine wind observations and is forced by sea surface temperature, sea ice cover, atmospheric composition changes, and solar forcing, and CERA-20C \citep{Laloyaux2018a}, which has an additional coupling to the ocean and was produced by a more recent version of the IFS model.

The other reanalyses are full-input products as they assimilate all available data, including satellite data \citep{Fujiwara2017}. The NCEP/NCAR Reanalysis I \citep[NR-1 --][]{Kalnay1996, Kistler2001} was the first global reanalysis, followed by the NCEP/DOE Reanalysis 2 \citep[NR-2 --][]{Kanamitsu2002} that fixed some identified problems. The Climate Forecast System Reanalysis \citep[CFSR --][]{Saha2010a} is the most recent reanalysis by NCEP. The Japanese 55-year Reanalysis \citep[JRA-55 --][]{Kobayashi2015, Harada2016} is produced by the Japan Meteorological Agency (JMA). Its version using conventional data only, JRA-55 Conventional \citep[JRA-55C --][]{Kobayashi2014}, was not considered in this work as it provides similar results as JRA-55 \citep{Horton2018b}. NASA's Global Modeling and Assimilation Office (GMAO) released the Modern-Era Retrospective Analysis for Research and Applications, version 2 \citep[MERRA-2 -- ][]{Gelaro2017}, which is an improvement of the first MERRA reanalysis \citep{Rienecker2011}. Finally, the predecessor of ERA5, ERA-Interim \citep[ERA-INT --][]{Dee2011a}, was also considered.

ERA5 \citep{Hersbach2019} is meant to replace ERA-Interim. It profits from multiple improvements to the Integrating Forecasting System (IFS), in terms of model physics, core dynamics, and data assimilation \citep{Hersbach2019}. It provides more outputs, with higher temporal (hourly) and spatial (31~km) resolutions. The use of a 10-member ensemble of data assimilations at a lower spatial and temporal resolution allows providing an estimate of the uncertainty. ERA5 assimilates significantly more data than ERA-Interim, such as ground-based radar and new satellite sensors, and uses improved observation operators, which allows to better compare model outputs and observations. Additionally, it indirectly profits from improvements of historical observations, both for conventional and satellite data \citep{Hersbach2019}. ERA5 is also more appropriate for climate analyses as it relies on suitable radiative forcing and boundary conditions (e.g. changes in greenhouse gases, aerosols, SST, and sea ice).


\begin{table}[bt]
	\caption{Assessed reanalysis datasets with their respective properties, sorted by type and model age.}
	\small
	\begin{threeparttable}
	\begin{tabular}{lllllll}
		\hline
		\headrow
		\thead{Name} & \thead{Institution} & \thead{Coverage} & \thead{Output} & \thead{Model resolution \& age} & \thead{Input} & \thead{Assimilation}\\
		\hline 
		\textbf{20CR-2c} & NOAA-CIRES & 1851 -- 2014 & 2\degree x 2\degree & T62 ($\sim$1.88\degree), L28, 2008 & surface  & EnKF\\
		\textbf{ERA-20C} & ECMWF & 1900 -- 2010 & 1\degree x 1\degree & TL159 ($\sim$1.13\degree), L9, 2012 & surface  & 4D-Var\\
		\textbf{CERA-20C} & ECMWF & 1901 -- 2010 & 1\degree x 1\degree & T159 ($\sim$1.13\degree), L91, 2016 & surface & 4D-Var\\
		\hline 
		\textbf{NR-1} & NCEP, NCAR & 1948 -- present & 2.5\degree x 2.5\degree & T62 ($\sim$1.88\degree), L28, 1995 & full & 3D-Var\\
		\textbf{NR-2} & NCEP, DOE & 1979 -- present & 2.5\degree x 2.5\degree & T62 ($\sim$1.88\degree), L28, 2001 & full  & 3D-Var\\
		\textbf{CFSR} & NCEP & 1979 -- present & 0.5\degree x 0.5\degree & T382 ($\sim$0.31\degree), L64, 2009 & full  & 3D-Var\\
		\textbf{JRA-55}  & JMA & 1958 -- present & 1.25\degree x 1.25\degree & TL319 ($\sim$0.36\degree), L60, 2009 & full  & 4D-Var\\
		%\textbf{JRA-55C}  & JMA & 1958 -- 2015 & 1.25\degree x 1.25\degree & TL319 ($\sim$0.36\degree), L60 & 2009 & conventional  & 4D-Var\\
		\textbf{MERRA-2} & NASA GMAO & 1980 -- present & 0.625\degree x 0.5\degree & 0.625\degree x 0.5\degree, L72, 2014 & full  & 3D-Var\\
		\textbf{ERA-INT} & ECMWF & 1979 -- 2017 & 0.75\degree x 0.75\degree & TL255 ($\sim$0.70\degree), L60, 2006 & full  & 4D-Var\\
		\textbf{ERA5} & ECMWF & 1950* -- present & 0.25\degree x 0.25\degree & T639 ($\sim$31~km), L137, 2016 & full  & 4D-Var\\
		\hline 
	\end{tabular} 

	\begin{tablenotes}
		\item *expected to be available in 2020. The available period starts in 1979 at time of writing.
		%\item JKL, just keep laughing; MN, merry noise.
	\end{tablenotes}
	\end{threeparttable}
	\label{table:datasets}
\end{table}

%ERA5 strengths compared to ERA-Interim
%Much higher spatial and temporal resolution
%Information on variation in quality over space and time
%Much improved troposphere
%Improved representation of tropical cyclones
%Better global balance of precipitation and evaporation
%Better precipitation over land in the deep tropics
%Better soil moisture
%More consistent sea surface temperature and sea ice

\subsection{Precipitation dataset}
\label{sec:precipitation}

The local variable to be predicted is here daily precipitation totals (06:00~h~UTC to 06:00~h~UTC the following day) at 301 stations of the MeteoSwiss network in Switzerland (Fig. \ref{fig:stations}), with a good coverage of the 1981--2010 period. The 30-year precipitation dataset was divided into a calibration period (CP) and an independent validation period (VP), which was evenly distributed over the entire series (1 year out of every 5; total of 6 years). The archive period (AP), where the analogue dates are being retrieved, is the same as the CP -- also with the VP being excluded -- with an additional exclusion of $\pm30$ days around the target date. Results are always presented for the VP.

\begin{figure}
	\includegraphics[width=70mm]{figures/map-stations.jpg}\\
	\caption{Map of the 301 precipitation stations with good data coverage of the period 1981--2010. Background map: \textcopyright\ SwissTopo.}
	\label{fig:stations}
\end{figure}


\subsection{Analog methods}
\label{sec:ams}

Most variants of the AM from \cite{Horton2018b} were considered (Table \ref{table:methods}). These AMs have different degrees of complexity, but they all start with a preselection in order to cope with seasonality. A commonly used preselection is based on the dates (PC: preselection on calendar basis in Table \ref{table:methods}) as candidate situations are extracted from the archive for a period of 120~days centered around the target date. To allow for a more dynamic approach, \citet{BenDaoud2016} based this preselection on the similarity of air temperature (T) at 925~hPa and 600~hPa at the nearest grid point. An undesired mixing of spring and autumn situations is discussed in \citet{Caillouet2016}. 

\begin{table*}[t]
	\caption{Analogue methods considered in the study, listed by increasing complexity. The analogy criterion is S1 for SLP and Z and RMSE for the other variables.}
	\small
	\begin{threeparttable}
		\begin{tabular}{llllll}
			\hline
			\headrow
			\thead{Method} & \thead{P0} & \thead{L1} & \thead{L2} & \thead{L3} & \thead{Reference} \\ 
			\hline 
			\multirow{2}{*}{\textbf{2Z}} & \multirow{2}{*}{PC} & Z1000@12h &&& \multirow{2}{*}{\citealp{Bontron2004}} \\
			&& Z500@24h &&& \\
			\hline 
			\multirow{4}{*}{\textbf{4Z}} & \multirow{4}{*}{PC} & Z1000@06h &&& \multirow{4}{*}{\citealp{Horton2018a}} \\
			&& Z1000@30h &&& \\
			&& Z700@24h &&& \\
			&& Z500@12h &&& \\
			\hline 
			\multirow{2}{*}{\textbf{2Z-2MI}} & \multirow{2}{*}{PC} & Z1000@12h & \multirow{2}{*}{MI850@12+24h} && \multirow{2}{*}{\citealp{Bontron2004}} \\
			&& Z500@24h &&& \\
			\hline 
			\multirow{4}{*}{\textbf{4Z-2MI}} & \multirow{4}{*}{PC} & Z1000@30h &&& \multirow{4}{*}{\citealp{Horton2018a}}\\
			&& Z850@12h & MI700@24h && \\
			&& Z700@24h & MI600@12h && \\
			&& Z400@12h &&& \\
			\hline 
			\multirow{2}{*}{\textbf{PT-2Z-4MI}} & T925@36h & Z1000@12h & MI925@12+24h && \multirow{2}{*}{\citealp{BenDaoud2016}} \\
			& T600@12h & Z500@24h & MI700@12+24h && \\
			\hline 
			\multirow{2}{*}{\textbf{PT-2Z-4W-4MI}} & T925@36h & Z1000@12h & \multirow{2}{*}{W850@06-24h} & MI925@12+24h & \multirow{2}{*}{\citealp{BenDaoud2016}} \\
			& T600@12h & Z500@24h && MI700@12+24h & \\
			\hline 
		\end{tabular} 
		
		\begin{tablenotes}
			\item P0, preselection (PC: $\pm 60$ days around the target date); L1, L2 and L3, subsequent levels of analogy.
			\item Z, geopotential height; T, air temperature; W, vertical velocity; MI, moisture index (product of the relative humidity at the given pressure level and the total water column).
		\end{tablenotes}
	\end{threeparttable}
	\label{table:methods}
\end{table*}

All considered AMs have a first level of analogy on the atmospheric circulation, using the geopotential height (Z) at different pressure levels and time as predictor. This analogy is quantified by the S1 criterion \citep[Eq.\ \ref{eq:S1}, ][]{Teweles1954, Brown2012}, which is a comparison of gradients over a selected (calibrated) domain. The objective is to compare the atmospheric flow and not the absolute values of the respective fields. The values of the criterion processed for different levels/hours are here averaged into a single value for a given candidate date. More advanced approaches introduce a weighting between these diverse components \citep{Horton2017a}.

\begin{equation}
	\label{eq:S1}
	S1=100 \frac{\sum_{i} \vert \Delta\hat{z}_{i} - \Delta z_{i} \vert}{\sum_{i} \max\left\lbrace \vert \Delta\hat{z}_{i} \vert; \vert \Delta z_{i} \vert \right\rbrace }
	% use \displaystyle to max sum sign larger
\end{equation}
where $\Delta \hat{z}_{i}$ is the geopotential height gradient between the \textit{i}-th pair of points for the target day, and $\Delta z_{i}$ is the corresponding observed geopotential height gradient for the candidate situation. The smaller the values S1 are, the more similar the pressure fields.

The most simple method, 2Z, is based only on the analogy of the atmospheric cirulation, using the geopotential height at 1000~hPa and 500~hPa, \citep{Bontron2004}. The most similar $N_{1}$ dates, with the lowest values of S1, are selected as analogues to the target date. The daily precipitation that was measured at these $N_{1}$ dates then provides the empirical conditional distribution, considered as the probabilistic prediction for the target date.

The second method, 4Z, is also based on geopotential height only, but using four combinations of pressure levels and temporal windows \citep{Horton2018a}. This method is here considered as a simplified version of a more complex method elaborated by genetic algorithms. It was found that using four geopotentiel height is more informative than using only two.

Then, the next three methods (2Z-2MI, 4Z-2MI, PT-2Z-4MI, Table \ref{table:methods}) add a second level of analogy to subsample from the first level, based on moisture variables. The predictor considered here is a moisture index (MI), which is the product of the total precipitable water (TPW) and the relative humidity (RH) and was introduced by \citet{Bontron2004}. This predictor is assessed using the root mean square error (RMSE) criterion. The 2Z-2MI method \citep{Bontron2004} considers this index at 850~hPa for two different hours (+12 h and +24 h). The 4Z-2MI method \citep{Horton2018a}, derived from optimizations with genetic algorithms, consider the 600~hPa and 700~hPa levels, but with a change in the selected levels of the geopotential height compared to 4Z. The 4Z-2MI method \citep{BenDaoud2016} considers the moisture index at 700~hPa and 925~hPa, both at +12 h and +24 h.

Finally, the most complex method, PT-2Z-4W-4MI, also named "SANDHY" for Stepwise Analogue Downscaling method for Hydrology \citep{BenDaoud2016, Caillouet2016}, adds an intermediate level of analogy, before the moisture predictors, based on the vertical velocity (W) at 850~hPa. It was primarily developed for large and relatively flat/lowland catchments in France (Sa\^{o}ne, Seine).

\subsection{Calibration of the methods}
\label{sec:calibration}

\section{Results}
\label{sec:results}

\subsection{Impact on the skill}
\label{sec:results_skill}

\begin{figure}[bt]
    \centering
    \includegraphics[width=\textwidth]{figures/boxplot-per-method.pdf}
    \caption{CRPSS for all stations, and for all considered AMs and reanalysis datasets on the VP. A higher CRPSS means better performance. The parameters of the AMs were calibrated for every station, every dataset, and every method. The boxes show the 25th, 50th, and 75th percentiles. The whiskers extend to the most extreme data point which is no more than 1.5 times the interquartile range.}
    \label{fig:comparison_values}
\end{figure}

\begin{figure}[bt]
    \centering
    \includegraphics[width=\textwidth]{figures/boxplot-per-method-diff.pdf}
    \caption{Impact of the reanalysis dataset on performance, isolated by processing the improvement in CRPSS for one dataset compared to the mean performance on all datasets, per station and per method. Note that the methods cannot be compared here, only the datasets. Same conventions as Fig. 2.}
    \label{fig:comparison_relative}
\end{figure}

%TODO: coeff correl?
%TODO: biases

\subsection{High resolution and its pitfalls}
\label{sec:results_hires}


\begin{figure}[bt]
	\centering
	\includegraphics[width=80mm]{figures/boxplot-resol.pdf}
	\caption{....}
	\label{fig:resolution}
\end{figure}


\subsection{Shared analog dates}
\label{sec:results_shared_dates}

\section{Conclusions}
\label{sec:conclusions}







%\subsubsection{Third Level Heading}
%\paragraph{Fourth Level Heading}
%\subparagraph{Fifth level heading}

% Here are examples of quotes and epigraphs.
%\begin{quote}
%The significant problems we have cannot be solved at the same level of thinking with which we created them.\endnote{Albert Einstein said this.}
%\end{quote}

%\begin{epigraph}{Albert Einstein}
%Anyone who has never made a mistake has never tried anything new.
%\end{epigraph}





\section*{acknowledgements}
...

\section*{conflict of interest}
...

%\printendnotes

% Submissions are not required to reflect the precise reference formatting of the journal (use of italics, bold etc.), however it is important that all key elements of each reference are included.
\bibliography{references}


\graphicalabstract{example-image-1x1}{Please check the journal's author guildines for whether a graphical abstract, key points, new findings, or other items are required for display in the Table of Contents.}

\end{document}
